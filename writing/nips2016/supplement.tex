\documentclass{article}

% to do (5/16)

%% put more sparkles around writing OED in a program. Why is this clever and useful?

%% introduce max Entropy prior early (could also be interpolation between max ent and the predictive)
%%%% results of sequence prediction (formerly, "randomness") w.r.t. maxEnt vs. predictive (discuss why we think maxEnt is giving a better prediction here)

%% Figure 3 histogram: vertical line with AIG
%% create n_subjects analysis figure (a la FIgure 3, line plot) for sequence prediction... bootstrap up to n=100 to observe crossover

% if you need to pass options to natbib, use, e.g.:
% \PassOptionsToPackage{numbers, compress}{natbib}
% before loading nips_2016
%
% to avoid loading the natbib package, add option nonatbib:

%\usepackage{nips_2016}

% to compile a camera-ready version, add the [final] option, e.g.:
\usepackage[final]{nips_2016}

\usepackage[utf8]{inputenc} % allow utf-8 input
\usepackage[T1]{fontenc}    % use 8-bit T1 fonts
%\usepackage{hyperref}       % hyperlinks
\usepackage{url}            % simple URL typesetting
\usepackage{booktabs}       % professional-quality tables
\usepackage{amsfonts}       % blackboard math symbols
\usepackage{nicefrac}       % compact symbols for 1/2, etc.
\usepackage{microtype}      % microtypography
\usepackage{amsmath}
\usepackage{mathtools}
\usepackage{fancyvrb}
\usepackage{multirow}
\usepackage{color}
\usepackage{textcomp}


% HT http://tex.stackexchange.com/a/151987/41154
\DeclarePairedDelimiterX{\infdivx}[2]{(}{)}{%
#1\;\delimsize\|\;#2%
}
\newcommand{\dkl}{D_\mathrm{KL}\infdivx}

\usepackage{listings}
\definecolor{lightgray}{rgb}{.9,.9,.9}
\definecolor{darkgray}{rgb}{.4,.4,.4}
\definecolor{purple}{rgb}{0.65, 0.12, 0.82}
\definecolor{orange}{rgb}{1,0.5,0}

\definecolor{Red}{RGB}{255,0,0}
\newcommand{\red}[1]{\textcolor{Red}{#1}}
\definecolor{Green}{RGB}{10,200,100}
\definecolor{Blue}{RGB}{10,100,200}
\newcommand{\ndg}[1]{\textcolor{Green}{[ndg: #1]}}
\newcommand{\mht}[1]{\textcolor{Blue}{[mht: #1]}}
\newcommand{\lou}[1]{\textcolor{orange}{[lou: #1]}}

% casual outlining font
\newcommand{\cas}[1]{ \textsf{\color{darkgray} \scriptsize #1} }

\lstdefinelanguage{JavaScript}{
keywords={typeof, new, true, false, catch, function, return, null, catch, switch, var, if, in, while, do, else, case, break},
keywordstyle=\color{blue}\bfseries,
ndkeywords={class, export, boolean, throw, implements, import, this},
ndkeywordstyle=\color{darkgray}\bfseries,
identifierstyle=\color{black},
sensitive=false,
comment=[l]{//},
morecomment=[s]{/*}{*/},
commentstyle=\color{purple}\ttfamily,
stringstyle=\color{red}\ttfamily,
morestring=[b]',
morestring=[b]"
}

\lstset{
language=JavaScript,
backgroundcolor=\color{white},
extendedchars=true,
basicstyle=\footnotesize\ttfamily,
showstringspaces=false,
showspaces=false,
numbers=none,
numberstyle=\footnotesize,
numbersep=9pt,
tabsize=2,
breaklines=true,
showtabs=false,
captionpos=b
}



\usepackage[ruled,vlined]{algorithm2e}

\newcommand{\ud}{\,\mathrm{d}}
\DeclareMathOperator*{\argmax}{arg\,max}


\title{Practical optimal experiment design with probabilistic programs: supplementary material}

% The \author macro works with any number of authors. There are two
% commands used to separate the names and addresses of multiple
% authors: \And and \AND.
%
% Using \And between authors leaves it to LaTeX to determine where to
% break the lines. Using \AND forces a line break at that point. So,
% if LaTeX puts 3 of 4 authors names on the first line, and the last
% on the second line, try using \AND instead of \And before the third
% author name.


\begin{document}
% \nipsfinalcopy is no longer used

\maketitle

\section{Sequence prediction}

\subsection{Markov model}

\begin{lstlisting}[upquote=true]
var markovCoin = function(seq) {
 Infer(function(){
   var transProb = uniformDraw(coinWeights)
   var sampleOne = function(lastFlip) {
       return flip(transProb) ? !lastFlip : lastFlip
   }
	  var sampleSeq = function(flipsSoFar, n) {
     if (n == 0) {
       return flipsSoFar
     } else {
       var nextFlip = sampleOne(last(flipsSoFar))
       return sampleSequence(append(flipsSoFar, nextFlip),
                             n - 1)
     }
   }
	  var sampledSeq = sampleSeq([flip(0.5)], seq.length - 1)
   condition(arrayEquals(seq, sampledSeq));
   return sampleOne(last(sampledSeq))
 })
}
\end{lstlisting}

\subsection{\texttt{groupify} linking function}

\begin{lstlisting}[upquote=true]
var groupify = function(model) {
  var groupified = function(x) {
    var sequence = x.sequence, n = x.n;
    var singleResponseDist = model(sequence);
    var p = Math.exp(score(singleResponseDist), true)
    return Binomial({n: n, p: p})
  }
  return groupified
}
\end{lstlisting}

\lstinline{groupify} is a higher order function that takes in a model of single-participant responses and returns an i.i.d. model of group responses.
We use the Bernoulli response probability defined by the single-subject model for a given sequence and use that to return a Binomial distribution on the number of heads responses from $n$ subjects.

\section{Category learning}

\subsection{Exemplar model}

Recall that the category learning models are classifiers that receive objects (vectors of boolean features) as input and return probability of A/B assignment as output.

The exemplar model defines a similarity function between two objects $s$ and $r$:
$$ \text{sim}(r, s) = \prod_i{\max(w_i, \delta_{s_i,r_i})  }$$
where $i$ denotes components of the objects, and $w_i$ are free parameters.
The model classifies an object $s$ as an A with probability:
$$ f_e(s ; w) = \frac{ \sum\limits_{r \in A}{\text{sim}(s,r)}}{
\sum\limits_{r \in A}{\text{sim}(s,r)} +
\sum\limits_{r \in B}{\text{sim}(s,r)}
 }$$
We express this model in WebPPL as:

\lstinputlisting[caption=Exemplar model in WebPPL, upquote=true]{exemplar.wppl}
\lstinline{exemplar} takes as input a set of As and Bs and returns as output the probability of A classification for each of the 16 objects in the experiment.
Because the model has continuous parameters, we integrate over them using 5000 samples of Metropolis-Hastings.

\subsection{Prototype model}

The original prototype model of Medin and Schaffer gave ordinal predictions; we consider a transformation of this model that gives continuous predictions.
Recall that the prototype model assumes people do not store information about all training examples but instead summary prototypes of A and B (these are computed from the training set by taking the mode).
The model defines a relative similarity function $h_i(s)$ that computes how much more similar $s$ is to the A prototype than it is to the B prototype along dimension $i$:
$$ h_i(s) =  w_i\left(2\frac{|\{r \in R ; r_i = s_i \} \cap A |}{|\{r \in R ; r_i = s_i \}|} - 1\right)$$
where $w_i$ are free parameters and $R = A \cup B$.
The model classifies an object $s$ as an A with probability:
$$ f_p(s ; w, \alpha, \beta) = \mathrm{logit}^{-1}\left\{ \alpha \times \left[ \beta |R \cap \{s\}| + \sum\limits_{i}{h_i(s)} \right] \right\}$$
where $\alpha$ is an extremizing parameter, and $\beta$ is a bias parameter for objects that have been encountered before. We express this model in WebPPL as:
\lstinputlisting[caption=Prototype model in WebPPL, upquote=true]{prototype.wppl}

\end{document}