% category learning models

Formally, the exemplar model classifies an object $s$ as an A with probability:
$$ f_e(s ; w) = \frac{ \sum\limits_{r \in A}{\sum\limits_{i}\max (w_i, \delta_{s_i, r_i})} }{
  \sum\limits_{r \in A}{\sum\limits_{i}\max (w_i, \delta_{s_i, r_i} )} + \sum\limits_{r \in B}{\sum\limits_{i}\max (w_i, \delta_{s_i, r_i})}} $$
where $w$ is a free parameter of dimensional weights.

The prototype model classifies an object $s$ as an A with probability:
$$ f_p(s ; w, \alpha, \beta) = \mathrm{logit}^{-1}\left\{ \alpha \times \left[ \beta |R \cap \{s\}| + \sum\limits_{i}w_i \left(2\frac{|\{r \in R ; r_i = s_i \} \cap A |}{|\{r \in R ; r_i = s_i \}|} - 1\right) \right] \right\}$$
where $w$ is a vector of weights, $\alpha$ is an extremizing parameter, and $\beta$ is a bias parameter for objects that have been encountered before.
\footnote{The original prototype model of Medin and Schaffer gave ordinal predictions; we consider a transformation of this model that gives continuous predictions.}
